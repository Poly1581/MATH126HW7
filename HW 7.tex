\documentclass[12pt]{amsart}
\usepackage{setspace}
\usepackage{fullpage}
\usepackage{cancel}
\usepackage{graphicx}
\def\R{{\hbox{\bf R}}}
\def\C{{\hbox{\bf C}}}
\def\Q{{\hbox{\bf Q}}}
\def\G{{\hbox{\bf G}}}
\def\P{{\hbox{\bf P}}}
\def\E{{\hbox{\bf E}}}
\def\V{{\hbox{\bf V}}}
\def\v{{\hbox{\bf v}}}
\def\f{{\overline{f}}}
\font \roman = cmr10 at 10 true pt
\def\Z{{\hbox{\bf Z}}}
\def\U{{\hbox{\bf U}}}
\def\A{{\hbox{\bf A}}}
\def\eps{\varepsilon}
\def\RZ{  {\mathbb R} \backslash {\mathbb Z} }
\def\Span{\hbox{ \rm Span} \,\,\, }
\def\ep{{\epsilon}}
\def\rfl {\rfloor}
\def\lfl {\lfloor}
\def\CF{{\hbox {\bf F}}}
\setlength{\parindent}{0pt}
\setlength{\parskip}{1cm plus4mm minus3mm}
%%%%%%%%%%%%%%%%%%%%%%%%%%%%%%%%%%%%%%%%%%%%%%%%%%%%%%%%%%%%%%%%%%%%%%%%%%%%%%%%%%%
%%%%%%%%%%%  LETTERS 
%%%%%%%%%%%%%%%%%%%%%%%%%%%%%%%%%%%%%%%%%%%%%%%%%%%%%%%%%%%%%%%%%%%%%%%%%%%%%%%%%%%

\newcommand{\barx}{{\bar x}}
\newcommand{\bary}{{\bar y}}
\newcommand{\barz}{{\bar z}}
\newcommand{\bart}{{\bar t}}

\newcommand{\bfP}{{\bf{P}}}

%%%%%%%%%%%%%%%%%%%%%%%%%%%%%%%%%%%%%%%%%%%%%%%%%%%%%%%%%%%%%%%%%%%%%%%%%%%%%%%%%%%
%%%%%%%%%%%%%%%%%%%%%%%%%%%%%%%%%%%%%%%%%%%%%%%%%%%%%%%%%%%%%%%%%%%%%%%%%%%%%%%%%%%
                                                                                
\newcommand{\parend}[1]{{\left( #1  \right) }}
\newcommand{\spparend}[1]{{\left(\, #1  \,\right) }}
\newcommand{\angled}[1]{{\left\langle #1  \right\rangle }}
\newcommand{\brackd}[1]{{\left[ #1  \right] }}
\newcommand{\spbrackd}[1]{{\left[\, #1  \,\right] }}
\newcommand{\braced}[1]{{\left\{ #1  \right\} }}
\newcommand{\leftbraced}[1]{{\left\{ #1  \right. }}
\newcommand{\floor}[1]{{\left\lfloor #1\right\rfloor}}
\newcommand{\ceiling}[1]{{\left\lceil #1\right\rceil}}
\newcommand{\barred}[1]{{\left|#1\right|}}
\newcommand{\doublebarred}[1]{{\left|\left|#1\right|\right|}}
\newcommand{\spaced}[1]{{\, #1\, }}
\newcommand{\suchthat}{{\spaced{|}}}
\newcommand{\numof}{{\sharp}}
\newcommand{\assign}{{\,\leftarrow\,}}
\newcommand{\myaccept}{{\mbox{\tiny accept}}}
\newcommand{\myreject}{{\mbox{\tiny reject}}}
\newcommand{\blanksymbol}{{\sqcup}}
                                                                                                                         
\newcommand{\veps}{{\varepsilon}}
\newcommand{\Sigmastar}{{\Sigma^\ast}}
                           
\newcommand{\half}{\mbox{$\frac{1}{2}$}}    
\newcommand{\threehalfs}{\mbox{$\frac{3}{2}$}}   
\newcommand{\domino}[2]{\left[\frac{#1}{#2}\right]}  

%%%%%%%%%%%% complexity classes

\newcommand{\PP}{\mathbb{P}}
\newcommand{\NP}{\mathbb{NP}}
\newcommand{\PSPACE}{\mathbb{PSPACE}}
\newcommand{\coNP}{\textrm{co}\mathbb{NP}}
\newcommand{\DLOG}{\mathbb{L}}
\newcommand{\NLOG}{\mathbb{NL}}
\newcommand{\NL}{\mathbb{NL}}

%%%%%%%%%%% decision problems

\newcommand{\PCP}{\sc{PCP}}
\newcommand{\Path}{\sc{Path}}
\newcommand{\GenGeo}{\sc{Generalized Geography}}

\newcommand{\malytm}{{\mbox{\tiny TM}}}
\newcommand{\malycfg}{{\mbox{\tiny CFG}}}
\newcommand{\Atm}{\mbox{\rm A}_\malytm}
\newcommand{\complAtm}{{\overline{\mbox{\rm A}}}_\malytm}
\newcommand{\AllCFG}{{\mbox{\sc All}}_\malycfg}
\newcommand{\complAllCFG}{{\overline{\mbox{\sc All}}}_\malycfg}
\newcommand{\complL}{{\bar L}}
\newcommand{\TQBF}{\mbox{\sc TQBF}}
\newcommand{\SAT}{\mbox{\sc SAT}}

%%%%%%%%%%%%%%%%%%%%%%%%%%%%%%%%%%%%%%%%%%%%%%%%%%%%%%%%%%%%%%%%%%%%%%%%%%%%%%%%%%%
%%%%%%%%%%%%%%% for homeworks
%%%%%%%%%%%%%%%%%%%%%%%%%%%%%%%%%%%%%%%%%%%%%%%%%%%%%%%%%%%%%%%%%%%%%%%%%%%%%%%%%%%

\newcommand{\student}[2]{%
{\noindent\Large{ \emph{#1} SID {#2} } \hfill} \vskip 0.1in}

\newcommand{\assignment}[1]{\medskip\centerline{\large\bf CS 111 ASSIGNMENT {#1}}}

\newcommand{\duedate}[1]{{\centerline{due {#1}\medskip}}}     

\newcounter{problemnumber}                                                                                 

\newenvironment{problem}{{\vskip 0.1in \noindent
              \bf Problem~\addtocounter{problemnumber}{1}\textbf{\arabic{problemnumber}:}}}{}

\newcounter{solutionnumber}

\newenvironment{solution}{{\vskip 0.1in \noindent
             \bf Solution~\addtocounter{solutionnumber}{1}\arabic{solutionnumber}:}}
				{\ \newline\smallskip\lineacross\smallskip}

\newcommand{\lineacross}{\noindent\mbox{}\hrulefill\mbox{}}

\newcommand{\decproblem}[3]{%
\medskip
\noindent
\begin{list}{\hfill}{\setlength{\labelsep}{0in}
                       \setlength{\topsep}{0in}
                       \setlength{\partopsep}{0in}
                       \setlength{\leftmargin}{0in}
                       \setlength{\listparindent}{0in}
                       \setlength{\labelwidth}{0.5in}
                       \setlength{\itemindent}{0in}
                       \setlength{\itemsep}{0in}
                     }
\item{{{\sc{#1}}:}}
                \begin{list}{\hfill}{\setlength{\labelsep}{0.1in}
                       \setlength{\topsep}{0in}
                       \setlength{\partopsep}{0in}
                       \setlength{\leftmargin}{0.5in}
                       \setlength{\labelwidth}{0.5in}
                       \setlength{\listparindent}{0in}
                       \setlength{\itemindent}{0in}
                       \setlength{\itemsep}{0in}
                       }
                \item{{\em Instance:\ }}{#2}
                \item{{\em Query:\ }}{#3}
                \end{list}
\end{list}
\medskip
}

%%%%%%%%%%%%%%%%%%%%%%%%%%%%%%%%%%%%%%%%%%%%%%%%%%%%%%%%%%%%%%%%%%%%%%%%%%%%%%%%%%%
%%%%%%%%%%%%% for quizzes
%%%%%%%%%%%%%%%%%%%%%%%%%%%%%%%%%%%%%%%%%%%%%%%%%%%%%%%%%%%%%%%%%%%%%%%%%%%%%%%%%%%

\newcommand{\quizheader}{ {\large NAME: \hskip 3in SID:\hfill}
                                \newline\lineacross \medskip }

%\newcommand{\namespace}{ {\large NAME: \hskip 3in SID:\hfill}
%                               \newline\lineacross \medskip }

%%%%%%%%%%%%%%%%%%%%%%%%%%%%%%%%%%%%%%%%%%%%%%%%%%%%%%%%%%%%%%%%%%%%%%%%%%%%%%%%%%%
%%%%%%%%%%%%% for final
%%%%%%%%%%%%%%%%%%%%%%%%%%%%%%%%%%%%%%%%%%%%%%%%%%%%%%%%%%%%%%%%%%%%%%%%%%%%%%%%%%%

\newcommand{\namespace}{\noindent{\Large NAME: \hfill SID:\hskip 1.5in\ }\\\medskip\noindent\mbox{}\hrulefill\mbox{}}


\begin{document}

\begin{center}
{\bf Math 126 HW 7, due 5/25}
\end{center}

%%%%%%%%%%%%%%%%%%%%%%%%%%%%%%%%%%%%%%%%%%%%%%%%%%%%%%%%%%%%%%%%%%%% UNINISHED %%%%%%%%%%%%%%%%%%%%%%%%%%%%%%%%%%%%%%%%%%%%%%%%%%%%%%%%%%%%%%%%%%%%%%%
%%%%%%%%%%%%%%%%%%%%%%%%%%%%%%%%%%%%%%%%%%%%%%%%%%%%%%%%%%%%%%%%%%%% PROBLEM  1 %%%%%%%%%%%%%%%%%%%%%%%%%%%%%%%%%%%%%%%%%%%%%%%%%%%%%%%%%%%%%%%%%%%%%%

\begin{problem}
\begin{itemize}
	\item[\textbf{1a:}] How many possible arrangements are there of the letters in the word STREETS?
	\item[\textbf{1b:}] How many of these arrangements have the two E's adjacent to eachother? (Hint: If you glue tthe two E's together, you are effectively only arranging $6$ letters instead of $7$)
	\item[\textbf{1c:}] How many of those arrangments have \textbf{no} two of the same letter adjacent?
\end{itemize}
\end{problem}
\begin{solution}
\begin{itemize}
	\item[\textbf{1a:}] There are three redundant (repeated) letters (S,T, and E).  We must compensate for them in our calculation of the possible arrangments.  The number of possible arrangements of these letters is $\frac{_7P_7}{(2!)^3} = \frac{5040}{2^3} = 630$.
	\item[\textbf{1b:}] We can construct one letter double E to replace both of the E's in the word and compute as if it were a proper letter.  This gives the number of arrangements of these letters as $\frac{_6P_6}{(2!)^2} = \frac{720}{4} = 180$.
	\item[\textbf{1c:}] This is slightly more complicated. We need to use the inclusion-exclusion principle and to do so, we need to calculate a few things (note, I will denote $A(S)$ as the arrangements where S is a list of potentially redundant letters that are required to be adjacent):
		\begin{itemize}
			\item{\textbf{Arrangements with no restrictions:}} $A(\emptyset) = \frac{_7P_7}{(2!)^3} = 630$
			\item{\textbf{$E^s$ adjacent:}} $A(E) = \frac{_6P_6}{(2!)^2} = 180$
			\item{\textbf{$T^s$ adjacent:}} $A(T) = 180$ (calculated in the same way as $A(E)$)
			\item{\textbf{$S^s$ adjacent:}} $A(S) = 180$ (calculated in the same way as $A(E)$)
			\item{\textbf{$E^s$ and $T^s$ adjacent:}} $A(E,T) = \frac{_5P_5}{(2!)^1} = 105$
			\item{\textbf{$E^s$ and $S^s$ adjacent:}} $A(E,S) = 105$ (calculated in the same way as $A(E,S)$)
			\item{\textbf{$T^s$ and $S^s$ adjacent:}} $A(T,S) = 105$ (calculated in the same way as $A(E,S)$)
			\item{\textbf{$E^s$ and $T^s$ and $S^s$ adjacent:}} $A(E,T,S) = \frac{_4P_4}{(2!)^0} = 24$
		\end{itemize}
		Now, finally, we can calculate the number of arragnements with no two same letters adjacent as $A(\emptyset) - A(E) - A(T) - A(S) + A(E,T) + A(E,S) + A(T,S) - A(E,T,S) = 630 - 3\cdot180 + 3\cdot105 - 24 = 381$.
\end{itemize}
\end{solution}

%%%%%%%%%%%%%%%%%%%%%%%%%%%%%%%%%%%%%%%%%%%%%%%%%%%%%%%%%%%%%%%%%%%% UNINISHED %%%%%%%%%%%%%%%%%%%%%%%%%%%%%%%%%%%%%%%%%%%%%%%%%%%%%%%%%%%%%%%%%%%%%%%
%%%%%%%%%%%%%%%%%%%%%%%%%%%%%%%%%%%%%%%%%%%%%%%%%%%%%%%%%%%%%%%%%%%% PROBLEM  2 %%%%%%%%%%%%%%%%%%%%%%%%%%%%%%%%%%%%%%%%%%%%%%%%%%%%%%%%%%%%%%%%%%%%%%

\begin{problem}

Recall that the chromatic polynomial of the hexagon is $c_{C_6}(k)=(k-1)^6+(k-1)$. Using this and the inclusion-exclusion priciple, determine how many proper colorings of the hexagon there are involving \textit{exactly} $4$ colors (i.e. each vertex is Red, Yellow, Blue, or Green. and each color is used at least once).
\end{problem}
\begin{solution}

\end{solution}

%%%%%%%%%%%%%%%%%%%%%%%%%%%%%%%%%%%%%%%%%%%%%%%%%%%%%%%%%%%%%%%%%%%% UNINISHED %%%%%%%%%%%%%%%%%%%%%%%%%%%%%%%%%%%%%%%%%%%%%%%%%%%%%%%%%%%%%%%%%%%%%%%
%%%%%%%%%%%%%%%%%%%%%%%%%%%%%%%%%%%%%%%%%%%%%%%%%%%%%%%%%%%%%%%%%%%% PROBLEM  3 %%%%%%%%%%%%%%%%%%%%%%%%%%%%%%%%%%%%%%%%%%%%%%%%%%%%%%%%%%%%%%%%%%%%%%

\begin{problem}
For each of the following sequences, find the corresponding generating function "in closed form" (this means that your final answer should not involve any infinite sums)
\begin{itemize}
	\item[\textbf{3a:}] $0,0,0,1,0,0,0,0,\dots$ (where $a_3=1$ but all other $a_j$ are zero.
	\item[\textbf{3b:}] $0,0,0,0,3,3,3,3,\dots$ (where $a_i=0 \forall i \leq 3 \text{ and } a_i=3 \forall i >3$)
	\item[\textbf{3c:}] $1,2,1,8,1,32,\dots$ (where $a_j=1$ for even $j$ and $2^j$ for odd $j$)
	\item[\textbf{3d:}] $1,1,1,1,1,1,5,1,1,\dots$ (where $a_6=5$, but all other $a_j$ are $1$)
	\item[\textbf{3e:}] $a,2,3,4,5,6,7,8,\dots$ (where $a_j=j+1$ for all $j$)
\end{itemize}
\end{problem}
\begin{solution}

\end{solution}

%%%%%%%%%%%%%%%%%%%%%%%%%%%%%%%%%%%%%%%%%%%%%%%%%%%%%%%%%%%%%%%%%%%% UNINISHED %%%%%%%%%%%%%%%%%%%%%%%%%%%%%%%%%%%%%%%%%%%%%%%%%%%%%%%%%%%%%%%%%%%%%%%
%%%%%%%%%%%%%%%%%%%%%%%%%%%%%%%%%%%%%%%%%%%%%%%%%%%%%%%%%%%%%%%%%%%% PROBLEM  4 %%%%%%%%%%%%%%%%%%%%%%%%%%%%%%%%%%%%%%%%%%%%%%%%%%%%%%%%%%%%%%%%%%%%%%

\begin{problem}

For each of the following combinatorial problems, give a function $G(x)$ and a value of $k$ such that the answer to the problem is the $x^k$ coefficient of $G(x)$ (e.g. if I had asked how many subsets of size $7$ there are of a set of size $10$, one answer would be "The $x^7$ coefficient of $(1+x)^{10}$".  You don't need to actually evaluate this coefficient!).
\begin{itemize}
	\item[\textbf{4a:}] I roll $10$ dice.  Five of the dice are numbered from $1$ to $6$, and the remaining five numbered from $1$ to $4$.  How many ways can I get a sum of $30$?
	\item[\textbf{4b:}] I roll $10$ dice.  Each die has $6$ sides, but the sides are numbered $1,2,3,4,5,5$ instead of $1,2,3,4,5,6$ (assume that the $5's$ are distinguishable -- i.e. that for each die there's $2$ ways it can come up $5$). How many ways can I get a sum of $30$?
	\item[\textbf{4c:}] I have a drawer containing $5$ identical red beads, $6$ identical blue beads, and $4$ identical green beads.  How many ways are there to select $8$ beads from the drawer (e.g. one way is to take $5$ red, $2$ blue, and $1$ green.  Order doesn't matter)?
	\item[\textbf{4d:}] I have a drawer containing $5$ identical red beads, $6$ identical blue beads, and $4$ identical green beads.  How many ways are there to select $8$ beads from the drawer if I must select at least two beads of each color?
	\item[\textbf{4e:}] I have a drawer containing $5$ identical red beads, $6$ identical blue beads, and $4$ identical green beads.  How many ways are there to select $8$ beads from the drawer if I must select an even number of each bead?
\end{itemize}
\end{problem}
\begin{solution}

\end{solution}

%%%%%%%%%%%%%%%%%%%%%%%%%%%%%%%%%%%%%%%%%%%%%%%%%%%%%%%%%%%%%%%%%%%% UNINISHED %%%%%%%%%%%%%%%%%%%%%%%%%%%%%%%%%%%%%%%%%%%%%%%%%%%%%%%%%%%%%%%%%%%%%%%
%%%%%%%%%%%%%%%%%%%%%%%%%%%%%%%%%%%%%%%%%%%%%%%%%%%%%%%%%%%%%%%%%%%% PROBLEM  5 %%%%%%%%%%%%%%%%%%%%%%%%%%%%%%%%%%%%%%%%%%%%%%%%%%%%%%%%%%%%%%%%%%%%%%

\begin{problem}

Let $b_n$ be the number of ways of choosing $n$ beads out of a drawer containing infinitely many red, yellow, blue, green, and orange beads such that
\begin{itemize}
	\item The number of red beads is even
	\item There is at most one blue bead
	\item The number of orange beads is either $0$ or $2$
\end{itemize}
(there is no restriction on yellow or green beads.)
\begin{itemize}
	\item[\textbf{5a:}] Compute the generating function for $b_n$ in closed form
	\item[\textbf{5b:}] Compute $b_{20}$
\end{itemize}
\end{problem}
\begin{solution}

\end{solution}

\end{document}


